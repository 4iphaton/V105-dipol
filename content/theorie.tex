\section{Zielsetzung}
\label{sec:Zielsetzung}

Das Ziel dieses Versuches ist es das magnetische Dipolmoment eines in eine
Billardkugel eingelassenen Stabmagneten über die Gravitation, die Pendeldauer
der schwingenden Kugel und die Präzessionsdauer der rotierenden Kugel zu bestimmen.


\section{Theorie}
\label{sec:Theorie}

Das zu bestimmende magnetische Moment $\mu_{dipol}$ des Stabmagneten
ist experimentell bestimmbar. Hierfür wird ausgenutzt, dass
sich ein Dipol solange in einem homogenen Magnetfeld durch ein
Drehmoment dreht bis $\vec{\mu}$ parallel zu $\vec{B}$ steht.
\begin{equation}
  \vec{M} = \vec{\mu} \times \vec{B} = \mu \, B \, sin(\theta_{1})
  \label{eqn:MB}
\end{equation}
Das benötigte homogene Magnetfeld wird durch ein Helmholtzspulenpaar erzeugt
und lässt sich nach
\begin{equation}
  B_0 = \frac{\mu_0I}{2}\frac{\symup{N}\symup{R}^2}{\left(\symup{R}^2+x^2\right)^{3/2}}
  \label{eqn:b0}
\end{equation}
berechnen. Wobei die Spule N Windungen, einen Radius von R und einen Abstand von d $=2x$
besitzt.

Zudem wird das Trägheitsmoment $\symbf{I}$ einer Kugel mit Masse M und Radius R nach
\begin{equation}
  \symbf{I}_\text{Kugel}=\frac{2}{5}\symup{M}\symup{R}^2
  \label{eqn:Ik}
\end{equation}
berechnet.

\subsection{Bestimmung über Gravitation}
Ein Gleichgewicht der Drehmomente der Gravitation nach
\begin{equation}
  {\vec{M}}_{g}= m \cdot \left(\vec{r}\times\vec{g}\right) = m\, r \, g\, sin(\theta_{2})
\end{equation}
und der homogenen magnetischen Flussdichte (\ref{eqn:MB})
für $\theta_1 =\theta_2$ wird erreicht und darüber
\begin{equation}
  \mu_\text{dipol}=\frac{mrg}{B}
  \label{eqn:grav}
\end{equation}
zu bestimmt.
%Hierbei ist m die kleine an der Aluminiumstange befestigte Masse und r der Abstand dieser Masse zur Kugel.
\subsection{Bestimmung über Schwingungsdauer}
Die Kugel in Schwingung zu versetzen lässt sie sich wie einen
harmonischen Oszillator verhalten. Die Lösung der Differentialgleichung
liefert
\begin{equation}
  \mu_\text{Dipol}=\frac{4\pi^2\symbf{I}_\text{Kugel}}{T^2B}
  \label{eqn:pendel}
\end{equation}
zum berechnen. Wobei $\symbf{I}_\text{Kugel}$ aus (\ref{eqn:Ik}) folgt
und T die Schwingungsdauer ist.

\subsection{Bestimmung über Präzessionsdauer}
Bei einwirken äußerer Kräfte beginnt die Drehachse um die
ursprüngliche Drehachse gemäß der Lösung der
Differentialgleichung
\begin{equation}
  \Omega_P = \frac{\mu\,B}{\vert{L_K}\vert}
\end{equation}
zu präzidieren. Mit dem Drehimpuls
\begin{equation}
  L_K = 2\pi\,\symbf{I}_\text{Kugel}\,\nu
\end{equation}
, wobei $\nu$ die Drehfrequenz und
 $\symbf{I}_\text{Kugel}$ das Trägheitsmoment aus (\ref{eqn:Ik}) ist.
 Daraus folgt zur Bestimmung des magnetischen Moments
 \begin{equation}
   \mu_\text{Dipol}=\frac{4\pi^2\symbf{I}_\text{Kugel}\,\nu}{TB}
   \label{eqn:strobo}.
 \end{equation}

\subsection{Fehlerformeln}
Die Mittelwerte werden durch
\begin{equation}
  \overline{x} =\frac{1}{N}\sum_{i=1}^{N} x_i
  \label{eqn:mean}
\end{equation}
bestimmt.
Zudem ist die Standardabweichung durch
\begin{equation}
  \sigma = \sqrt{\left(\frac{1}{N}\sum_{i=1}^{N}(x_i-\overline{x})^2\right)}
  \label{eqn:std}
\end{equation}
 gegeben.
 Zuletzt ergibt sich die Fehlerfortpflanzung aus
 \begin{equation}
  \increment y = \sqrt{\sum_{i=1}^{n}\left(\frac{\partial y}{\partial x_i}
  \increment x_i\right)^2}
  \label{eqn:gauss}.
 \end{equation}
