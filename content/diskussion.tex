\section{Diskussion}
\label{sec:Diskussion}

Die Werte der drei Messungen $\mu_\text{gravitation}=$ \SI{0.803(004)}{\ampere\square\meter}, $\mu_\text{Schwingung}$ $=$ \SI{1.008(018)}{\ampere\square\meter}
und $\mu_\text{präzession}=$ \SI{0.950(6)}{\ampere\square\meter} liegen mit höchstens \SI{25.53}{\percent} Abweichung (Abweichung von $\mu_\text{Schwingung}$ zu $\mu_\text{gravitation}$)
relativ nah beieinander und geben somit eine sinnvolle Größenordnung für das Dipolmoment an, umfassen sich mit ihren Fehlern jedoch nicht gegenseitig.
Der Wert mit der höchsten Abweichung zu den beiden anderen Werten und somit wahrscheinlich auch der höchsten Abweichung zum eigentlichen Wert ist $\mu_\text{gravitation}$.
Dies ist dadurch zu begründen, dass die Schwebelage des Gewichts für den absoluten Grenzfall nicht erkennbar ist, sodass für einen erkennbaren Effekt eine leicht höhere Feldstärke angelegt
werden muss, wodurch die Messwerte verfälscht werden. Auch wenn der Wert für die Präzession mittig liegt und somit am besten scheint, muss hier eine Verfälschung der Messung
aufgrund der Nutationsbewegungen angenommen werden, durch welche dieses Dipolmoment ebenfalls etwas zu klein ist. Die Messung mit den wenigsten
möglichen Fehlerquellen ist die Bestimmung per Pendeldauer, da hier lediglich die Ablesefehler mit einspielen, welche durch die Mittlung über 10 Perioden so klein wie
möglich gehalten werden.
