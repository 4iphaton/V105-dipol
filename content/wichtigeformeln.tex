\section{Folgende Formeln werden benötigt}
Feld in der Mitte der Helmholzspule mit N Windungen, Radius R und Abstand d $=2x$:
\begin{equation}
  B_0 = \frac{\mu_0I}{2}\frac{\symup{N}\symup{R}^2}{\left(\symup{R}^2+x^2\right)^{3/2}}
  \label{eqn:b0}
\end{equation}

Trägheitsmoment $\symbf{I}$ einer Kugel mit Masse M und Radius R:
\begin{equation}
  \symbf{I}_\text{Kugel}=\frac{2}{5}\symup{M}\symup{R}^2
  \label{eqn:Ik}
\end{equation}

Die Mittelwerte werden durch
\begin{equation}
  \overline{x} =\frac{1}{N}\sum_{i=1}^{N} x_i
  \label{eqn:mean}
\end{equation}
bestimmt.

Die Standardabweichung wird durch
\begin{equation}
  \sigma = \sqrt{\left(\frac{1}{N}\sum_{i=1}^{N}(x_i-\overline{x})^2\right)}
  \label{eqn:std}
\end{equation}
 errechnet.

Zudem ergibt sich die Fehlerfortpflanzung aus
\begin{equation}
 \increment y = \sqrt{\sum_{i=1}^{n}\left(\frac{\partial y}{\partial x_i}
 \increment x_i\right)^2}
 \label{eqn:gauss}
\end{equation}

Bestimmung über Gravitation
\begin{equation}
  \mu_\text{dipol}=\frac{mrg}{B}
  \label{eqn:grav}
\end{equation}

Bestimmung über Periodendauer des Pendels
\begin{equation}
  \mu_\text{Dipol}=\frac{4\pi^2\symbf{I}_\text{Kugel}}{T^2B}
  \label{eqn:pendel}
\end{equation}

Bestimmung über Präzessionsdauer bei Drehfrequenz $\nu$
\begin{equation}
  \mu_\text{Dipol}=\frac{4\pi^2\symbf{I}_\text{Kugel}*\nu}{TB}
  \label{eqn:strobo}
\end{equation}
